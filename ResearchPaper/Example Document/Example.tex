% From LaTeX: It's Not Just for Academia, Part 2, Kevin O'Malley
% O'Reilly MacDevCenter, Feb 2004
%=-=-=-=-=-=-=-=-=-=-=-=-=-=-=-=-=-=-=-=-=-=-=-=-=-=-=-=-=-=-=-=-=-=
% Begin document preamble.
%=-=-=-=-=-=-=-=-=-=-=-=-=-=-=-=-=-=-=-=-=-=-=-=-=-=-=-=-=-=-=-=-=-=
\documentclass{article}
\usepackage{color,graphicx}
\usepackage{url}

\setlength{\textheight}{9.45in}
\setlength{\oddsidemargin}{-0.5in}
\setlength{\evensidemargin}{-0.5in}
\setlength{\topmargin}{-0.49in}
\setlength{\footskip}{0.5in}
\def \columnsep{0.2in}
\def \textwidth{7.45in}

%=-=-=-=-=-=-=-=-=-=-=-=-=-=-=-=-=-=-=-=-=-=-=-=-=-=-=-=-=-=-=-=-=-=
% Start of the document (end of preamble).
%=-=-=-=-=-=-=-=-=-=-=-=-=-=-=-=-=-=-=-=-=-=-=-=-=-=-=-=-=-=-=-=-=-=
\begin{document}

\title{\vspace{-0.75in} LaTeX, it's not just for Academia - Part 2} 
\author{Kevin O'Malley}
\maketitle

%=-=-=-=-=-=-=-=-=-=-=-=-=-=-=-=-=-=-=-=-=-=-=-=-=-=-=-=-=-=-=-=-=-=
% Table of contents.
%=-=-=-=-=-=-=-=-=-=-=-=-=-=-=-=-=-=-=-=-=-=-=-=-=-=-=-=-=-=-=-=-=-=
\tableofcontents
\pagebreak
 
 \textbf{Note - This is an early draft of the article. See the O'Reilly MacDevCenter site for the final version.}
 
%=-=-=-=-=-=-=-=-=-=-=-=-=-=-=-=-=-=-=-=-=-=-=-=-=-=-=-=-=-=-=-=-=-=
\section {Introduction}
%=-=-=-=-=-=-=-=-=-=-=-=-=-=-=-=-=-=-=-=-=-=-=-=-=-=-=-=-=-=-=-=-=-=
In the first article of this two-part series, you learned about
LaTeX's most popular implementations for Mac OS X, and that it's not a
word processor, but rather a document preparation and typesetting
system. Under Mac OS X, there are many high-quality writing
environments that simplify the process of composing LaTeX documents.

Before going any further, let me address some corrections/omissions
from the first article. As a few readers pointed out, TeXShop and
OzTeX can be setup to use Fink's UNIX TeX package. Saying "X11-based
LaTeX implementations" may have mislead readers. TeX and X11 are not
coupled in any way. You use X11-based DVI or PDF previewers to view
LaTeX output. The file II2.dmg is not from the TeXShop folks, but
rather is linked from their page. The program is available from the
i-Installer Home Page. Finally, many readers have asked for more
complete LaTeX sources. I have included these in the Resources
section.

Now, let's see how you can use LaTeX to accomplish some common writing
tasks. As you can imagine, LaTeX has many commands and features, and
can be somewhat complicated to learn. Today, I'll keep things simple
and concentrate on the basics.


This article introduces you to the LaTeX language, the basic structure
of a LaTeX document, and shows some common writing tasks and how to
accomplish them using LaTeX. Finally, I will demonstrate a few
examples of common LaTeX documents and show how to generate these
documents in different output formats, including formatting your
documents for the web.

For these examples, you can use any LaTeX environment you wish -
TeXShop, OzTeX, or the commandline version. At this point it may be
helpful to briefly discuss the environment I use. I use TeXShop if I
wish to use a Mac OS X Aqua-based program. Other programs have more
feature, but I like the simplicity and stability of TeXShop.

For writing, I use GNU Emacs (21.2.1 from Fink) under Apple's X11. I
do not rely on document previewing much, but when I need it, I do the
following. I use a Perl script that looks at my LaTeX document every
few seconds and calls an output generation program when the document
changes; when I save the LaTeX document, the preview is automatically
generated. I also have the script generate and display the current
word count for the document.

Recently, I have been using Enrico Franconi Enhanced Carbon Emacs
[http://www.inf.unibz.it/~franconi/mac-emacs]. This implementation was
pointed out by a reader in the Trackback section. This Emacs
implementation is based on the emacs-21.3.50 CVS distribution.
Overall, I really like this version, which includes LaTeX tools such
as AucTeX and RefTeX. In addition, it has good integration with
Aqua-based previewing programs like Acrobat Reader and TeXShop.


The script generates output in PDF. To view the PDF file, I use
TeXShop's external editor feature, which permits TeXShop to redisplay
the PDF when it has changed.

I also use the Emacs' Flyspell mode to automatically check spelling as
I type and suggests corrections.

Another way to generate your output is by using a Makefile.

Take a look at the Perl script and Makefile. These will give you
examples of how to process a LaTeX document and generate various
output formats.

%=-=-=-=-=-=-=-=-=-=-=-=-=-=-=-=-=-=-=-=-=-=-=-=-=-=-=-=-=-=-=-=-=-=
\section{LaTeX Preliminaries - Commands, Comments, and Whitespace}
%=-=-=-=-=-=-=-=-=-=-=-=-=-=-=-=-=-=-=-=-=-=-=-=-=-=-=-=-=-=-=-=-=-=
Before we begin composing a document, it's helpful to know a bit about how LaTeX handles some common
document elements. Like most programming languages, LaTeX commands are case sensitive. A LaTeX
command is formatted as follows:

\begin{verbatim}
\command-name[optional-parameters]{parameter}
\end{verbatim}

A LaTeX command begins with a backslash followed by the command
name. Commands can have parameters, which are enclosed by curly braces
({}) as well as optional parameters, which are surrounded by square
brackets ([]).

Any line beginning with the \% character is a comment. When LaTeX
reads the \% character in a file, it treats the rest of that line as a
comment. In addition to the comment character, you can use the comment
command.

\begin{verbatim}
\begin{comment}
This is a comment and is not displayed in the final document.
\end{comment}
\end{verbatim}

%=-=-=-=-=-=-=-=-=-=-=-=-=-=-=-=-=-=-=-=-=-=-=-=-=-=-=-=-=-=-=-=-=-=
\section{LaTeX Document Structure}
%=-=-=-=-=-=-=-=-=-=-=-=-=-=-=-=-=-=-=-=-=-=-=-=-=-=-=-=-=-=-=-=-=-=
Now that you know some basics, lets move on to the structure of a
LaTeX document. When you program in C++, Python, or Java, you usually
structure source files in a certain way. Same holds for LaTeX. Here is
the format of a LaTeX document.

\begin{verbatim}
% -- Begin LaTeX document.
% Begin document preamble.
\documentclass[options]{document-class-name}
\usepackage{package-name}
...
% End document preamble.
\begin{document}
% -- Your text and LaTeX commands go here.
\end{document}
% -- End LaTeX document.
end{verbatim}

A LaTeX document begins with the document class command. This command
specifies the class of your document, and consequently, what commands
and environments are supported and how LaTeX formats the
document. LaTeX document classes include article, report, and book, to
name a few. Think of the document class as defining the type of
writing you are doing as well as its appearance. For example, if you
are going to write an article, use the article class; a book, use the
book class.

Here are the most common LaTeX document classes. 

\begin{itemize}
\item letter - for writing letters
\item article - for writing short reports, papers, and documentation
\item report - for larger writing tasks such as large reports having
  several chapters, or a theses
\item book - for writing books
\item slides - for preparing slides
\end{itemize}

The options parameter is used to change the settings for a document
class. For example, most classes slides, accepts the typeface size
options 10pt, 11pt, 12pt - 10pt is the default. Another option is the
paper size, which includes a4paper, a5paper, b5paper, letterpaper,
legalpaper, executivepaper - letter is the default.  You can also
specify the documents page orientation. For example, landscape tells
LaTeX to display the document in landscape mode.

Here are some example of common document classes.

\begin{verbatim}
\documentclass{letter}
\documentclass[11pt]{report}
\documentclass[12pt,landscape] {article}
\documentclass{slide}
\end{verbatim} 

You add extra functionally and commands through packages. Many
packages come with LaTeX such as color, graphicx, and makeidx. In
addition, you can add packages written by others, or even write your
own for specific writing needs. Following the package command, you can
define other commands that alter the formatting of your document.

The section of your document from document class to the begin document
command is called the Document Preamble. The document preamble
contains statements, packages, and other commands that set values and
change the appearance of your document. LaTeX documents end with the
end document command. Any text after this command is ignored.

The begin document command signals the beginning of your actual
writing. Between the begin and end document commands is where you add
your text and various LaTeX commands. What commands you can use is
determined by the document class and the packages you include.

Before concluding this discussion, let's look at one more thing - the
environment command. The environment command enables you to control
the formatting and appearance of your document's content. For example,
imagine you need to create a list of text, add an abstract, or display
a table. For these cases, LaTeX provides environments that make it
easy to accomplish these goals. You can even define your own
environments, or redefine existing envoronments, with the new
environment command.

%=-=-=-=-=-=-=-=-=-=-=-=-=-=-=-=-=-=-=-=-=-=-=-=-=-=-=-=-=-=-=-=-=-=
\section{Common LaTeX Operations}
%=-=-=-=-=-=-=-=-=-=-=-=-=-=-=-=-=-=-=-=-=-=-=-=-=-=-=-=-=-=-=-=-=-=
No mater what kind of document you write, there are always common
operations that you need to perform. For example, when writing a
report or an article, you need to divide your writing into sections,
add tables, include footnotes, or possibly include graphics.

This section shows you how to perform many common operations in
LaTeX. These suggestions are in no way inclusive, or enumerate all of
the possibilities. The goal here is to get you started with some
simple versions of commands. This discussion should be augmented with
a good LaTeX reference.

The output for each example is displayed in a single output
document. As you read about these commands, its a good idea to take a
look at the example document's source file [Insert link to
examples.tex] and output file [Insert link to examples.pdf]. Also, see
the LaTeX source file for this article [Insert link to
mac\_osx\_latex.tex].

%-------------------------------------------------------------------
\subsection{Sections}
%-------------------------------------------------------------------
LaTeX sections provide a mechanism for structuring your
document. LaTeX supports several commands that you use to organize
your document into sections.

\begin{itemize}
\item part
\item chapter
\item section
\item paragraph
\item subparagraph
\end{itemize}

To see sections in action, see the example and source files
for this article.

%-------------------------------------------------------------------
\subsection{Including Files}
%-------------------------------------------------------------------
When working on large documents, it can be difficult to maneuver and
edit a single file that contains all your text and LaTeX
commands. Fortunately, LaTeX provides a mechanism for breaking large
documents into smaller pieces. To accomplish this, you use the include
and input commands.

The include command generates a page break and inserts the named file
at the location of the command. The input command also inserts the
file, but with no page break. The named file does not include the file
extension, just the file name. For example, to include the file
chapter.tex you would just specify the name chapter.

\begin{verbatim}
\include{file-name}
\input{file-name}
\end{verbatim} 

There are certain rules for doing this so consult the LaTeX
documentation for more information.

%-------------------------------------------------------------------
\subsection{Page Numbers}
%-------------------------------------------------------------------
You control the appearance of page numbers with the page numbering
command. LaTeX offers several options for numbering pages, including
arabic, roman, Roman, alpha, and Alpha.

\begin{verbatim}
\pagenumbering{roman}
\end{verbatim} 

Page numbers are one of several commands for changing the appearance
of pages. You can also change the margins, number of columns, add
headers and footers to a page, and create a title page and
abstract. See the Page Styles section of a LaTeX reference for more
information.

%-------------------------------------------------------------------
\subsection{Table of Contents/List of Figures/List of Tables}
%-------------------------------------------------------------------
You create a table of contents in your document with the table of
contents command. This command places a table of contents at the
location of the command. LaTeX generates the table of contents entries
from the section commands of the document.

LaTeX will automatically handle the numbering of sections and chapters
for you. You can suppress adding sections to the table of contents, as
well as section numbers, by using the starred version of the command.

\begin{verbatim}
\subsection*{My Title}
\end{verbatim}

This is also useful if you are writing an article with no table of
contents, and do not want numbers to appear for each section.

The list of figures and the list of tables command places a list of
all figures and tables in your document. The list is inserted at the
location of the command. If the caption command appears in the figure
environment, that figure is added to the list of figures. If the
caption command appears in the table environment, the table is added
to the table list. The text of the caption is what's displayed in
figure or table list.

To generate the initial list of figures, tables, and table of
contents, and need to run the LaTeX command twice.

\begin{verbatim}
% latex latex-file    
% latex latex-file
\end{verbatim} 

%-------------------------------------------------------------------
\subsection{Generating an Index}
%-------------------------------------------------------------------
If you are writing a reference document, a book, or a report, it's
sometimes useful to include an index section at the end of the
document. An index is a list of key words in the document and the
page, or pages, where they occur. You add an index to your document
with the makeinx package.

To create an index, you first include and enable the makeinx package
in the document preamble. Next, add the index command to each word you
wish to include in the index.

\begin{verbatim}
...
\usepackage{makeidx}
\makeindex
...
\begin{document}
As you learned in the first article, in LaTeX\index{LaTeX}, you can manage your bibliographic citations using BibTeX\index{BibTeX}.
...
\end{document}
end{verbatim} 

Finally, you generate the index by running the following
programs (do not add the extension, just the file name).

\begin{verbatim}
% latex latex-file    
% makeindex latex-file
% latex latex-file

or

% pdflatex latex-file    
% makeindex latex-file
% pdflatex latex-file
\end{verbatim} 

Here's how it works. First, LaTeX processes the file and writes each
index entry to a file, with a .idx extension. Next, makeindex
creates the sorted index, which it stores in a file with a .ind
extension. Finally, you run LaTeX again to place the index into the
document.

You can create many different index formats by changing the syntax of
the index command. See a LaTeX reference for more information.

%-------------------------------------------------------------------
\subsection{BibTeX}
%-------------------------------------------------------------------
As you learned in the first article, you manage your bibliographic
citations using BibTeX. BibTeX is a separate program that you use to
manage your bibliographic citations and merge selected citation
information onto your LaTeX documents.

BibTeX solves some very common problems that arise when constructing a
bibliography. For example, imagine you are writing a paper that
includes many references and citations. Further imagine that the
publication you are submitting to requires citation information to be
formatted in a certain format; such as ACM or IEEE. Without BibTeX,
you would need to maintain a list of your references and format each
entry by hand in the required format. If you need to submit the paper to
another publication, with a different citation format, you have to
reformat each citation by hand.

BibTeX solves this problem quite easily. Here's what you need to
do. First, maintain your citations in a format that BibTeX
understands. You do this by formatting each citation in a defined
format and placing it into a file with a .bib extension. You can place
all citations in a single file, or break them up as you wish.

Here's an example of the format:

\begin{verbatim}
@misc{ wilson-ltxx,
  author = "Peter R. Wilson",
  title = "LTX2X: A LATEX to X Auto-tagger",
  url = "citeseer.nj.nec.com/wilson97ltxx.html" }
\end{verbatim} 

BibTeX identifies each record by its key. In the above example, the
key is wilson-ltxx. This is the value you use to reference the
citation within your LaTeX document.

As you can imagine, the format has many options and parameters. See
the Resource section for more information. In fact, you can get many
citations already formatted in BibTeX format at sites such as CiteSeer
[http://citeseer.ist.psu.edu].

To create a citation in your text, use the cite command, with the
parameter specifying the citation key. This command places a citation
mark into the final text pointing to the reference and includes the
reference in the bibliography. You can add a reference to the
bibliography, without adding a reference mark to the main text with
the no cite command.

To create a bibliography, add the following commands just before
the end document command.

\begin{verbatim}
...
\bibliographystyle{plain}
\bibliography{sources}
...
\end{document}
\end{verbatim} 
 
The bibliography style command specifies the bibliographic style used
to format each citation. The bibliography command instructs LaTeX to
insert the appropriate citations into the document. BibTeX supports
many bibliographic styles. See its documentation for more information.

To generate your bibliography, perform the following steps.

\begin{itemize}
\item Run LaTeX on the LaTeX file
\item Run BibTeX on the LaTeX file
\item Run LaTeX again on the LaTeX file
\item Run LaTeX yet again on the LaTeX file
\end{itemize} 

\begin{verbatim}
% latex latexfile
% bibtex latex-file
% latex latex-file
% latex latex-file
\end{verbatim} 

%-------------------------------------------------------------------
\subsection{Footnotes}
%-------------------------------------------------------------------
LaTeX supports the inclusion of footnotes in your text with the
footnote command. The footnote command takes an optional argument,
num, that you can use to change the default footnote number.

\begin{verbatim}
\footnote{Text for the footnote.}
\end{verbatim} 

In addition to the footnote command, you can also include footnotes
with the footnote mark and footnote test commands. The footnote mark
and footnote test commands are used in concert to place footnotes at
the botton of a page.

%-------------------------------------------------------------------
\subsection{Tables}
%-------------------------------------------------------------------
Table are something that we all use in our documents. LaTeX enables
you to create high-quality tables through the table and tabular
environments. Using these environments, you can easily produce very
well-structured and readable tables of information. The basic syntax
for creating a table is as follows:

\begin{verbatim}
\begin{table}[where]
\caption{Table caption}
\centering
\begin{tabular}[pos]{cols}
 column 1 & column 2 ... & column k \\
 ...
\end{tabular}
\end{table}
\end{verbatim} 

Where LaTeX places a table is very important. For example, if you have
a large table in your document, you would rather see it on a single
page, rather than broken up across pages. In the LaTeX literature, you
will see the term "float'" or "floating object" to refer to this
idea. This describes the situation where a table or figure can not fit
on its current page, and is placed on a separate so-called floating
page.

LaTeX enables control over the placement of tables through the where
parameter. The where parameter defines where the table is displayed on
the page. A value of b places the table at the bottom of the page, h
places the table here, t at the top of the page, and p on a separate
float page containing no text, only floats.

The concept of floating objects also applied to figures and footnotes.

You use the tabular environment to construct the table. The pos and
cols parameters control how the table is formatted. The pos parameter
controls the vertical position of the whole tabular environment. The
values are either t (align with top row) or b (align with bottom
row). The cols parameter controls the column formatting; l = format
text left, r = format text right, c = format text center. The p\{wd\}
parameter controls the size of a column and makes columns with
multi lines.

%-------------------------------------------------------------------
\subsection{Figures}
%-------------------------------------------------------------------
You add figures to a document with the figure command. Like tables,
this places a figure at the location of the command, conditioned by
the where parameter. Figures can also float over pages so make sure
you read the description of floating objects in the Tables section.

You insert a figure as follows:

\begin{verbatim}
\begin{figure}[where]
\caption{Figure caption}
...
\end{figure}
\end{verbatim} 

%-------------------------------------------------------------------
\subsection{Graphics}
%-------------------------------------------------------------------
Most documents are composed of not only text, but also graphics. YOu
insert graphics into a LaTeX file with the insert graphic
command. However, you can not insert a JPG or PDF file into a LaTeX
document, only EPS files. To insert other file formats like JPG, you
have a few choices. One way is to convert your graphic file to EPS and
then insert it into the document.


To convert the file you can use a Mac OS X Aqua-based programs, such
as Graphics Converter, or install ImageMagick from Fink and use the
following command:


\begin{verbatim}
% convert [jpeg-file] [eps-file]
\end{verbatim} 

You insert a graphics file into a document by first using the graphicx
package and then using the insert graphics command..

\begin{verbatim}
\usepackage{graphicx}
..
\begin{document}
...
\includegraphics{mjh.eps}
...
\end{verbatim} 

To insert a JPG file directly, use the PDFLaTeX program to process
your LaTeX document.

\begin{verbatim}
\usepackage{graphicx}
..
\begin{document}
...
\includegraphics{mjh.jpg}
...
\end{verbatim} 

This explanation is quite basic and there are far more issues to
working with graphics in LaTeX. In addition to inserting graphics,
LaTeX supports commands that enable you to draw picture elements such
as boxes, trees, and curves in your document. Consult the Resource
sections for more information.

%-------------------------------------------------------------------
\subsection{Math}
%-------------------------------------------------------------------
One of the major reasons for using LaTeX is its strong handling of
mathematical equations and formulas. This is a complex subject so I'm
just going to touch on the basics to give you a feel for how it
works, and looks.

You insert mathematical symbols into your document by either
surrounding the expressions with the \$ symbol, or using the
begin/end math commands.

The time-points $x_{i}$ to $x_{j}$ are embedded in the main text.


\begin{math}
  (l_{1} \leq x_{j} - x_{i} \leq u_{i}) \vee ... \vee (l_{n} \leq x_{j} - x_{i} \leq u_{n})
\end{math}


%-------------------------------------------------------------------
\subsection{Lists}
%-------------------------------------------------------------------
There are many times when you need to include a list of information in
a document. To create lists, LaTeX provides the itemize, enumerate,
and description environments. Each of these environments enable you to
easily add different types of lists to your documents. All of these
lists can be nested so you can produce various kinds of lists with
different sub-list levels.

\begin{verbatim}
\begin{list-environment}
\item item 1 text
\item item 2 text
\end{list-environment}
\end{verbatim} 

%=-=-=-=-=-=-=-=-=-=-=-=-=-=-=-=-=-=-=-=-=-=-=-=-=-=-=-=-=-=-=-=-=-=
\section{Examples}
%=-=-=-=-=-=-=-=-=-=-=-=-=-=-=-=-=-=-=-=-=-=-=-=-=-=-=-=-=-=-=-=-=-=
Now that you understand some basic LaTeX commands, lets put it all
together by looking at some sample LaTeX documents. We will look at
two kinds of documents; an article and a slide presentation.

For these example, I will provide a sample document source file that
you can use as a starting point for creating your own documents. Be
advised - these are simple examples that contain minimal commands.

%-------------------------------------------------------------------
\subsection{An Article}
%-------------------------------------------------------------------
Composing papers and reports is something many of us do daily. For
example, if you are a student, you need to take class notes and write
papers for your classes. If you are a developer, you can write many
kinds of project documents such as design documents and reference
manuals. 

Imagine you need to write a short paper for a class or maybe a series
of articles. For this
project you use the article document class.

See the LaTeX source file for this article [Insert link to
mac\_osx\_latex.tex].

%=-=-=-=-=-=-=-=-=-=-=-=-=-=-=-=-=-=-=-=-=-=-=-=-=-=-=-=-=-=-=-=-=-=
\subsection{A Presentations}
%=-=-=-=-=-=-=-=-=-=-=-=-=-=-=-=-=-=-=-=-=-=-=-=-=-=-=-=-=-=-=-=-=-=
At one point or another, we have all had to give a
presentation. Faced with this, most of us will reach for PowerPoint or
Apple's Keynote. Both programs are excellent choices that enable you
to quickly build very professional presentations. However, you can
also create slides and computer-based presentations with LaTeX.

There are many advantages to using LaTeX for your
presentations. First, LaTeX presentations require less disk space to
store than their PowerPoint or Keynote counterparts. With disks being
cheep these days, this is not much of an issue. But, I still like the
idea of keeping file sizes down when I can.

Another reason for choosing LaTeX is if your presentation uses lots of
mathematical equations.. If you use LaTeX, you can easily copy and
paste these into your slides. Otherwise, you need to some way of
getting them out of LaTeX and into your presentation software (maybe
using tex2im or TeXShop).

Finally there's the simplicity factor. I don't know about you, but the
world we live in has become preoccupied with appearance at the expense
of content. I find it distracting to go to a talk and watch cute
animations and graphics when plan old text would have been a lot
clearer. Yes, a picture is worth a thousand words, but the picture
does not need to fly on to the screen for the slide to be effective.

With LaTeX, there are many ways to create quality slides. The
simplest is to use the LaTeX slides class, which comes with your LaTeX
distribution. In the past, if you wanted to create slides you used
SLiTeX. Today, this has been replace by the slide class. The slides
class enables you to create simple slides with very little work.

There are many other ways to creating slides with LaTeX. Some of the
most popular are FoilTeX, Prosper, Seminar, ifmslide, pdfslide,
PPower4.

I create my presentations using a combination of PDFLaTeX, FoilTeX,
and PPower4. My needs are quite basic, as you will see in the example
slides [Insert link to slides.tex]. For a more complete solution, see
``Creating Presentations in PDFLaTeX'' by Matt Welsh (see the
Reference sections).

%=-=-=-=-=-=-=-=-=-=-=-=-=-=-=-=-=-=-=-=-=-=-=-=-=-=-=-=-=-=-=-=-=-=
\section{Output Formats}
%=-=-=-=-=-=-=-=-=-=-=-=-=-=-=-=-=-=-=-=-=-=-=-=-=-=-=-=-=-=-=-=-=-=
As you learned in the first article, a great reason for using LaTeX is
that it can render your source document in many output formats,
including PDF, Postscript, HTML, and RTF.

%-------------------------------------------------------------------
\subsection{Putting Your Documents on the Web}
%-------------------------------------------------------------------
These days, putting documents on the web is a common task. For
example, imagine keeping notes in LaTeX and generating HTML so you can
view them from anywhere. The best way to accomplish this is to
install latex2html by Nikos Drakos (available from Fink). Once you
have it installed, all you need to do is type the following command to
generate a HTML version of your document.

\begin{verbatim}
% latex2html latex-file
\end{verbatim} 

The latex2html Perl script processes the specified LaTeX file,
converts it to a set of HTML pages, and places the HTML files into a
new directory, titled with the name of the file. There are lots of
options for this script. One useful option is split (-split [n]). If n
= -1, latex2html generates one large HTML file. If n == some number
then that many pages are generated.

Here is an example of the HTML pages [Insert link to mac\_osx\_latex
directory] generated from this article.

%-------------------------------------------------------------------
\subsection{Converting to RTF}
%-------------------------------------------------------------------
The latex2rtf program (available from Fink) converts a LaTeX file into
RTF format. This provides some compatibility with users of other
programs, such as MS-Word. In truth, the compatibility it affords is
minimal and is not very useful for larger, more complex documents.

%-------------------------------------------------------------------
\subsection{Converting to Text}
%-------------------------------------------------------------------
In the first article, you were introduced to a program called
DeTeX. DeTeX is a filter that removes LaTeX/TeX control sequences from
its input. You use DeTeX to extract your text from a LaTeX document. 
DeTex has several options so be sure to read its man page.

Another option is to convert the document to a single HTML file, using
latex2html. Then, open the HTML file in a web browser and save it as
text.

%-------------------------------------------------------------------
\subsection{Converting to MS-Word}
%-------------------------------------------------------------------
An often asked question is can how I take a document I created in LaTeX
and convert it to Word. The short answer is that I have not found any
reliable way to do this. There are some strategies you can use such as
converting the LaTeX document to text (using detex) and importing it
into Word, but all formatting is lost. Another is to convert to RTF,
and import the RTF to Word; but, this will also lose some formatting
information. Another is to convert to HTML and then import to Word,
but again, you will lose formatting information.

If anyone has discovered a reliable way to handle this, please post it
using Talkback.

%=-=-=-=-=-=-=-=-=-=-=-=-=-=-=-=-=-=-=-=-=-=-=-=-=-=-=-=-=-=-=-=-=-=
\section{LaTeX Integration with Mac OS X Applications}
%=-=-=-=-=-=-=-=-=-=-=-=-=-=-=-=-=-=-=-=-=-=-=-=-=-=-=-=-=-=-=-=-=-=
Many Mac OS X applications support integration with LaTeX. As you saw
in the first article, BBEdit supports a user-contributed glossaries of
LaTeX commands. OmniOutliner
(http://www.omnigroup.com/applications/omnioutliner/) from The Omni
Group is a Mac OS X application for creating many different kinds of
lists and outlines. OmniOutliner LaTeX Export
(http://www.opendarwin.org/~landonf/software/Omni-LaTeX/) is a set of
AppleScripts by Landon Fuller that export an OmniOutliner document to
LaTeX format.

%=-=-=-=-=-=-=-=-=-=-=-=-=-=-=-=-=-=-=-=-=-=-=-=-=-=-=-=-=-=-=-=-=-=
\section{Conclusion}
%=-=-=-=-=-=-=-=-=-=-=-=-=-=-=-=-=-=-=-=-=-=-=-=-=-=-=-=-=-=-=-=-=-=
I hope you've enjoyed this brief tour of LaTeX on Mac OS X. As you've
seen, Mac OS X supports a rich set of LaTeX environments and
implementations, making it a formidable platform for composing LaTeX
documents. If you use a word processor for your writing, using LaTeX
will be a shift in focus, and will probably require some time to
learn. But, as these articles have pointed out, there are many
advantages to using LaTeX.

In any event, I hope these articles have inspired you to begin using,
or return to using, LaTeX. Good luck and as always, if you discover
something interesting, post it using Talkback or send me an email.

\end{document}